%%%%%%%%%%%%%%%%%%%%%%%%%%%%%%%%%%%%%%%%%%%%%%%%%%%%%%%%%%%%%%%
% IMPORTANTE: Cambiar el compilador a XeLaTeX en las opciones %
%          Plantilla ECI basada en U Pontificia Chile         %
%               Creditos Github: @diegocostares               %
%%%%%%%%%%%%%%%%%%%%%%%%%%%%%%%%%%%%%%%%%%%%%%%%%%%%%%%%%%%%%%%
\documentclass{style}
\addbibresource{bibliography/references.bib}

\begin{document}
%%%%%%%%%%%%%%%%% RENOMBRE %%%%%%%%%%%%%%%%%
\graphicspath{ {./img/} }
\renewcommand{\contentsname}{Índice de contenido}
\renewcommand{\listfigurename}{Índice de figuras}
\renewcommand{\listtablename}{Índice de Tablas}
\renewcommand{\tablename}{Tabla}
\renewcommand{\figurename}{Imagen}
\renewcommand*{\lstlistingname}{Código}


%%%%%%%%%%%%%%%%% ENCABEZADO %%%%%%%%%%%%%%%%%
% Si se quiere eliminar, se tiene que quitar tambien las configuraciones del style.cls
\fancyhead[L]{ % Encabezado a la izquierda
  \begin{picture}(0,0) \put(0,30){\includegraphics[width=25mm]{img/logo_eci.png}} \end{picture}
  \put(75,40){\textcolor{gray}{\scriptsize{\begin{tabular}{l}
    ESCUELA COLOMBIANA DE INGENIERÍA JULIO GARAVITO\\
    DEPARTAMENTO DE INGENIERÍA DE SISTEMAS
  \end{tabular}}}}
}
% Si se quiere agregar a la derecha remplazar:
\rhead{} % Opcion 1
%\rhead{\begin{picture}(0,0) \put(-85,12){\includegraphics[width=30mm]{img/IMAGEN.png}} \end{picture}} % Opcion 2


%%%%%%%%%%%%%%%%% PORTADA %%%%%%%%%%%%%%%%%
% Opcion 1: Importar un pdf tamaño carta de EEUU
%\includepdf[pages=-]{img/Portada.pdf} % cambiar portada por pdf
% Opcion 2: Escribir una portada completa
\begin{titlepage}
\includegraphics[width=2.5cm]{img/logo_eci_h.png} % AGREGAMOS EL LOGO UC
\vspace*{-2.2cm} % Esta linea va con la anterior


\begin{tabular}{l}
\hspace{2.5cm} ESCUELA COLOMBIANA DE INGENIERÍA JULIO GARAVITO\\
\hspace{2.5cm} DEPARTAMENTO DE INGENIERÍA DE SISTEMAS\\
\end{tabular}

\begin{center}
\vspace{3cm}
{\scshape\Huge \textbf{Nombre del proyecto} \par}
\rule{80mm}{0.1mm}
%\vspace{1cm}

{\itshape\LARGE Proyecto de curso PTIA \par}
\vfill
{\Large \textbf{Grupo XX} \par}
\vspace{0.5cm}
{\large \textbf{Integrantes:}\\\normalsize Nombre1, Nombre2, Nombre \par} 


\medskip
\textbf{Fecha de entrega:} <día> de <mes> de <año>
\vspace{3cm}
\end{center}
\end{titlepage}
\newpage
% Opcion 3: Sin portada - Se recomienda quitar indices

%%%%%%%%%%%%%%%%% PRELIMINARES %%%%%%%%%%%%%%%%

% El siguiente comando es usado para crear las cajas de firmas
\newcommand*\wildcard[2][6cm]{\vspace{2cm}\parbox{#1}{\hrulefill\par#2}} 

\section*{Declaración firmada}

\vspace{1cm}

“Declaro que he escrito este trabajo de investigación por mí mismo, y que no he utilizado otras fuentes o recursos que los indicados para su preparación. Declaro que he indicado claramente todas las citas directas e indirectas, y que este documento no se ha presentado en otro lugar para fines de examen o publicación"

\vspace{3cm}
\textbf{Nombre del estudiante \& Firma:}
\vspace{1cm}

\begingroup

    \begin{center}
        \wildcard{\centerline{<Nombre estudiante>} ~\\ \centerline{Fecha: <día> de <mes> de <año>}}
        \hspace{2cm} % "hspace{}"  para espacio horizontal entre misma fila
        \wildcard{\centerline{<Nombre estudiante>} ~\\ \centerline{Fecha: <día> de <mes> de <año>} }
    \end{center}

\endgroup


\pagebreak

%%%%%%%%%%%%%%%%% INDICES %%%%%%%%%%%%%%%%%
\setstretch{1}
\tableofcontents
\listoftables
\listoffigures
%\lstlistoflistings % Para hacer indice de codigo

%%%%%%%%%%%%%%%%% ESPACIADO %%%%%%%%%%%%%%%%%
\setstretch{1.15} 


%%%%%%%%%%%%%%%%% CONTENIDO %%%%%%%%%%%%%%%%%
\newpage
%\input{content/tutorial} % HABILITAR Y DESHABILITAR PARA TENER EJEMPLOS
\section{Introducción}

\begin{tcolorbox}[colback=yellow!10!white,colframe=red!75!black,title=Recomendaciones]
  Definición de los problemas
  \begin{itemize}
      \item ¿Cuáles serían tres problemas que les interesaría resolver?
      \item ¿Por qué como proyecto de PTIA?
  \end{itemize}
\end{tcolorbox}

\pagebreak
\section{Trabajos relacionados}

\begin{tcolorbox}[colback=yellow!10!white,colframe=red!75!black,title=Recomendaciones]
  Definir el estado del arte
  \begin{itemize}
      \item ¿Cuáles son dos trabajos relacionados a cada problema? (resultados y tecnología)
  \end{itemize}
\end{tcolorbox}

\pagebreak
\section{Descripción del problema}

\begin{tcolorbox}[colback=yellow!10!white,colframe=red!75!black,title=Recomendaciones]
  \begin{itemize}
      \item ¿Cuáles criterios van a usar para seleccionar el problema a resolver?
      \item ¿Cuál es el problema seleccionado?
  \end{itemize}
\end{tcolorbox}

\pagebreak
\input{content/justificacion}
\section{Alcance y Objetivos}

\begin{tcolorbox}[colback=yellow!10!white,colframe=red!75!black,title=Recomendaciones]
  \begin{itemize}
      \item ¿Cuál es el problema que efectivamente van a resolver?
      \item ¿Por qué consideran que es una simplificación adecuada?
      \item ¿Cuál sería el objetivo (cuantitativo y cualitativo)? Detallar objetivo general
  \end{itemize}
\end{tcolorbox}

\pagebreak
\section{Diseño metodológico}

\begin{tcolorbox}[colback=yellow!10!white,colframe=red!75!black,title=Recomendaciones]
Definir el paso a paso como se planear dar solución al problema
  \begin{itemize}
      \item ¿Cuál es la estrategia general seleccionada para la solución? Justificación
      \item ¿Dónde se concentra la inteligencia de la solución? (conocimiento+razonamiento) 
      \item ¿Qué herramienta seleccionaron para implementar la solución? Justificación
      \item ¿Cuál es la arquitectura de la solución? (Componentes y relaciones)
      \item ¿Cuál es el componente principal? (Descripción detallada)
  \end{itemize}
\end{tcolorbox}

\pagebreak
\section{Análisis de resultados}

\begin{tcolorbox}[colback=yellow!10!white,colframe=red!75!black,title=Recomendaciones]
Detallar los resultados
  \begin{itemize}
      \item ¿Cuáles son los dos mejores casos de prueba? (Secuencia de entradas y salidas esperadas y logradas)
      \item ¿Cuál es el caso de prueba más significativo? (Además de la secuencia de entradas y salidas, analicen el resultado)
  \end{itemize}
\end{tcolorbox}

\pagebreak
\section{Análisis de resultados}

\begin{tcolorbox}[colback=yellow!10!white,colframe=red!75!black,title=Recomendaciones]
Presentar una reflexión sobre los resultados y hallazgos de su trabajo; lecciones aprendidas
  \begin{itemize}
      \item ¿Cuáles son los dos aciertos y dos errores más relevantes? (Aprendizaje de cada uno de ellos)
      \item ¿Cuáles son las tres conclusiones más evidentes?
      \item ¿Cuál es una buena idea de trabajo futuro?
  \end{itemize}
  
  \textit{Las conclusiones son una oportunidad para resumir los resultados y hallazgos de una investigación o un proyecto, y para hacer énfasis en la idea o punto principal del escrito al final de su argumento}
\end{tcolorbox}

\pagebreak


%%%%%%%%%%%%%%%%% BIBLIOGRAFIA %%%%%%%%%%%%%%%%%
\newpage
\printbibliography

%%%%%%%%%%%%%%%%%%% ANEXOS %%%%%%%%%%%%%%%%%%%%%

\newpage
%\input{anexos/anexo_LaTeX} % HABILITAR Y DESHABILITAR PARA TENER EJEMPLOS

\end{document}